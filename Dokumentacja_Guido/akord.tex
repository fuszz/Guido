\documentclass[dokumentacja.tex]{subfiles}

\begin{document}
\section{Akord}

\subsection{Zależności}
\paragraph*{Klasa \texttt{`Tonacja'} importuje:}
\begin{itemize}
    \item Klasę \texttt{dzwiek}
    \item Klasę \texttt{tonacja}
\end{itemize}
\paragraph*{Oraz wykorzystuje typy wyliczeniowe:}
\begin{itemize}
    \item \texttt{WartosciNut}
    \item \texttt{Funkcje}
    \item \texttt{Przewroty}
\end{itemize}


\subsection{Pola}
\begin{itemize}
    \item \texttt{\_dlugosc: WartosciNut} - przechowuje wartości nut (długość akordu jest jednakowa dla wszstkich głosów)
    \item \texttt{\_sopran: Dzwiek} - przechowują dźwięki odpowiednio dla każdego z głosów
    \item \texttt{\_alt: Dzwiek} - j.w.
    \item \texttt{\_tenor: Dzwiek} - j.w.
    \item \texttt{\_bas: Dzwiek} - j.w.
\end{itemize}

\subsection{Metody}
\begin{itemize}
    \item Konstruktor parametryczny
        \begin{python}
    def __init__(self, nowy_sopran: dzwiek.Dzwiek, nowy_alt: dzwiek.Dzwiek, nowy_tenor: dzwiek.Dzwiek, nowy_bas: dzwiek.Dzwiek, dlugosc: float):
        \end{python}
        Konstruktor przyporządkowuje odpowiednie dźwięki do odpowiednich pól. W przypadku, gdy \texttt{dlugosc} przyjmuje wartość spoza Enuma, zwracana jest \texttt{Funkcja.BLAD}.

    \item Akcesory dźwięków:
          \begin{python}
    def podaj_sopran(self) -> dzwiek.Dzwiek:
    def podaj_alt(self) -> dzwiek.Dzwiek:
    def podaj_tenor(self) -> dzwiek.Dzwiek:
    def podaj_bas(self) -> dzwiek.Dzwiek:    
          \end{python}

    \item Metoda zwracająca funkcję danego akordu w odniesieniu do pewnej konkretnej tonacji:
        \begin{python}
     def ustal_funkcje(self, dana_tonacja: tonacja.Tonacja) -> funkcje.Funkcja:
        \end{python}
        Metoda zwraca jedną z dopuszczalnych wartości typu wyliczeniowego Funkcja. Jeżeli dźwięki nie tworzą żadnej sensownej w danej tonacji funkcji, podnoszony jest \texttt{}

    \item Metoda zwracająca przewrót akordu:
    \begin{python}
    def ustal_przewrot(self, dana_tonacja: tonacja.Tonacja) -> przewroty.Przewrot:
    \end{python}
    Metoda zwraca jedną z dopuszczalnych wartości typu wyliczeniowego przewrot.Przewrot, przy czym wywołuje ona w swoim ciele metodę \texttt{ustal\_funkcje}, by poznać funkcję. 
    Jeżeli funkcja jest niepoprawna, niemożliwe jest określenie przewrotu akordu, w związku z czym zwraca się \texttt{Przewrot.NIE\_ZDEFINIOWANO}.

    \item Metoda zwracająca informację, który składnik akordu jest zdwojony:
    \begin{python}
    def ustal_dwojenie(self, dana_tonacja: tonacja.Tonacja) -> zdwojony_skladnik.ZdwojonySkladnik:
    \end{python}
    Metoda zwraca informację, który ze składników akordu jest zdwojony i zwraca informację o tym w typie wyliczeniowym ZdwojonySkladnik.
\end{itemize}
\end{document}

