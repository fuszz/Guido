\documentclass[dokumentacja.tex]{subfiles}
\usepackage{xcolor}
\lstset{
    basicstyle=\ttfamily\small,
    keywordstyle=\color{blue},
    morekeywords={metrum:, tonacja:, takty:},
    alsoletter={T,:},
    morekeywords=[2]{T, :},
    keywordstyle=[2]\color{red},
}

\begin{document}
\section{Obsługa plików wejściowych}
\subsection{Format plików wejściowych}
Na potrzeby projektu, ustalono format danych wejściowych. Dane wejściowe są plikami tekstowymi (.txt). Pliki wejściowe dzielą się na nagłówek i ciało. 
\paragraph*{Nagłówek}
\begin{itemize}
    \item metrum - \texttt{metrum: [3/4, 4/4]}
    \item liczbę taktów - \texttt{takty: [liczba naturalna]}
    \item tonację - \texttt{tonacja: [tonacje durowe i molowe z koła kwintowego]}
\end{itemize}
\paragraph*{Ciało}
W ciele pliku podaje się akordy. Akord jest uporządkowaną czwórką danych:
\begin{itemize}
    \item dźwięk sopranu - para danych: dźwięk i cyfra symbolizująca oktawę, przy czym zamiast -is i -es korzystamy odpowiednio ze znaku \# i b. 
    \item dźwięk altu
    \item dźwięk tenoru
    \item dźwięk basu
    \item wartość akordu - podana jako liczba zmiennoprzecinkowa
\end{itemize} 
Po podaniu wszystkich akordów mających znaleźć się w takcie występuje wielka litera "T" w nowej linii. 
W pliku wejściowym niedopuszczalna jest pusta linia, w której nie znajduje się żadna informacja. Będzie skutkować to błędem i paniką programu. 
\begin{lstlisting}[caption={Przykładowy poprawny plik wejściowy}, label=lst:wejscie-przyklad]
    metrum: 4/4
    takty: 4
    tonacja: C
    c4, e3, g2, c1, 2.0
    f3, c3, a2, f1, 1.0
    g3, d3, h2, g1, 1.0
    T
    c4, e3, g2, c1, 2.0
    f3, c3, a2, f1, 1.0
    g3, d3, h2, g1, 1.0
    T
    c4, e3, g2, c1, 2.0
    f3, c3, a2, f1, 1.0
    g3, d3, h2, g1, 1.0
    T
    c4, e3, g2, c1, 2.0
    f3, c3, a2, f1, 1.0
    g3, d3, h2, g1, 1.0
    T
    \end{lstlisting}

\subsection{Obsługa wczytywania plików}
Do wczytania pliku wejściowego w powyższym formacie służy plik \texttt{`obsluga\_pliku.py'}, w którym znajdują się trzy funkcje:
\begin{itemize}
    \item Funkcja wywoływana w celu odczytania pliku. Wywołuje dwie poniższe metody. \begin{python}
def odczytuj_plik(sciezka_do_pliku: str) -> partytura.Partytura:
    \end{python}
    \item Funkcja, która odczytuje nagłówek pliku \begin{python}
def utworz_partyture(plik: TextIO) -> partytura.Partytura:
    \end{python}
    \item Funkcja, która wypełnia nowoutworzony obiekt akordami\begin{python}
def wypelnij_partyture_akordami(plik: TextIO, nowa_partytura: partytura.Partytura) -> partytura.Partytura:
    \end{python} 
\end{itemize}

\subsection{Możliwe wyjątki}
\begin{itemize}
    \item \texttt{`BladWNaglowku'} - podnoszony przez funkcję \texttt{utworz\_partyture()}, gdy błąd zwraca funkcja \texttt{utworz\_partyture()} lub kiedy nie utworzono partytury
    \item \texttt{`BladWCiele'} - podnoszony przez funkcję \texttt{wypelnij\_partyture\_akordami()}, gdy błąd jest zwracany przez funkcje \texttt{}
    \item \texttt{`ValueError("Niepoprawne wczytanie danych")'} - zwracany przez funkcję \texttt{odczytuj\_plik}, kiedy wystąpił któryś z powyższych wyjątków. 
\end{itemize}

\end{document}