\documentclass[dokumentacja.tex]{subfiles}

\begin{document}
\section{Dźwięk}

\subsection{Zależności}
\paragraph*{Klasa \texttt{`Dzwiek'} importuje:}
\begin{itemize}
    \item Klasę \texttt{'Tonacja'}
\end{itemize}
\paragraph*{Klasa wykorzystuje typy wyliczeniowe:}
\begin{itemize}
    \item \texttt{'NazwyDzwiekow'}, który służy do czuwania nad poprawnym nazewnictwem dźwięków
    \item \texttt{`BezwzgledneKodyDzwiekow'}, używany przez metodę \texttt{Dzwiek.podaj\_swoj\_kod\_bezwzgledny()}.
\end{itemize}

\subsection{Pola}
\begin{itemize}
    \item \texttt{\_nazwa\_dzwieku: NazwyDzwiekow} - przechowuje wartość typu Enum "NazwyDzwiekow"
    \item \texttt{\_oktawa\_dzwieku: bool} - przechowuje oktawę, do której dźwięk przynależy. Poszczególnym oktawom przyporządkowano następujące numery:
    \begin{itemize}[label = o]
        \item 0 - oktawa subkontra
        \item 1 - oktawa kontra
        \item 2 - oktawa wielka
        \item 3 - oktawa mała 
        \item 4 - oktawa razkreślna
        \item 5 - oktawa dwukreślna
        \item 6 - oktawa trzykreślna
        \item 7 - oktawa czterokreślna
        \item 8 - oktawa pięciokreślna
    \end{itemize}
\end{itemize}

\subsection{Metody}
\begin{itemize}
    \item Konstruktor parametryczny
        \begin{python}
     def __init__(self, nowa_oktawa_dzwieku: int, nowa_nazwa_dzwieku: str):
        \end{python}
        gdzie \texttt{nowa\_oktawa\_dzwieku} jest przypisywana do pola \texttt{\_oktawa\_dzwieku}, a \texttt{nowa\_nazwa\_dzwieku} służy do poprawnego wyboru odpowiedniej wartości z enum \texttt{NazwyDzwiekow}. Jeśli w Enum brak odpowiedniej wartości, konstruktor zwraca \texttt{ValueError}.

    \item Akcesor nazwy dźwieku:
          \begin{python}
    def podaj_nazwe_dzwieku(self) -> str:
          \end{python}
        Zwraca wartość enuma, jest to typ \texttt{string}.  

    \item Metoda zwracająca stopień dźwięku w pewnej tonacji:
        \begin{python}
    def podaj_swoj_stopien(self, odpytywana_tonacja: tonacja.Tonacja) -> int:
        \end{python}
        gdzie \texttt{odpytywana\_tonacja} jest typu \texttt{`Tonacja'} i w zależności od niej podajemy stopień. Należy pamiętać, że dla prymy tonacji (np. dźwięk 'c' w tonacji 'C-dur') metoda zwróci 0, a dla septyma (np. dźwięk 'h' w tonacji 'C-dur') metoda zwróci 6.


    \item Metoda zwracająca względny kod dźwięku w tonacji:
    \begin{python}
    def podaj_swoj_kod_wzgledny(self, odpytywana_tonacja: tonacja.Tonacja) -> int:
    \end{python}
    Względny kod dźwięku jest wyliczany jako:
        \begin{equation}
            \textit{względny kod dźwięku} = \textit{numer oktawy} \cdot 7 + \textit{stopień dźwięku w tonacji}
        \end{equation}
        \clearpage
        \item Metoda zwracająca bezwzględny kod dźwięku w tonacji: \label{punkt:bezwzglednyKodDzwieku}

        \begin{python}
        def podaj_swoj_kod_bezwzgledny(self) -> int:
        \end{python}
        \nopagebreak
        Bezwzględny kod dźwięku jest niezależny od tonacji i reprezentuje pewną wysokośc dźwięku. Rekurencyjny wzór na bezwzględny kod dźwięku możemy wyrazić jako:
        \[
            \begin{cases}
                \textit{x } \sharp = \textit{x} + 1 \\
                \textit{x } \flat= \textit{x} - 1 \\
                \textit{x} = \textit{oktawa(x)} \cdot 12 + \textit{kod(x)}
            \end{cases}
            \]
            gdzie:
            \[
            \begin{cases}
                \textit{kod(x)} = 0; \textit{  x} = \textit{dźwięk c} \\
                \textit{kod(x)} = 2; \textit{  x} = \textit{dźwięk d} \\
                \textit{kod(x)} = 4; \textit{  x} = \textit{dźwięk e} \\
                \textit{kod(x)} = 5; \textit{  x} = \textit{dźwięk f} \\
                \textit{kod(x)} = 7; \textit{  x} = \textit{dźwięk g} \\
                \textit{kod(x)} = 9; \textit{  x} = \textit{dźwięk a} \\
                \textit{kod(x)} = 11; \textit{  x} = \textit{dźwięk h} \\
            \end{cases}
            \]
            oraz
            \begin{equation}
                \textit{oktawa(x)} \text{ jest numerem oktawy, w której dźwięk się znajduje}
            \end{equation}
            
    
\end{itemize}
\end{document}
