\documentclass[dokumentacja.tex]{subfiles}

\begin{document}
\section{Typy wyliczeniowe wykorzystane w bibliotece}
\subsection{\texttt{`NazwyDzwiekow'}}
Elementami typu są wszystkie dźwięki gamowłaściwe występujące w tonacjach koła kwintowego. Nazwami elementów są nazwy dźwięków w języku polskim, a wartościami są oznaczenia przyjęte na potrzeby projektu (każde podwyższenie o pół tonu oznaczamy jako `$\sharp$`, każde obniżenie to `$\flat$`). Są to:
\begin{center}
\begin{tabular}{|c|c|}
    \hline
    \textbf{Nazwa dźwięku w języku polskim} & \textbf{Kod na potrzeby projektu} \\
    \hline
    ces & c$\flat$\\ \hline
    c & c \\ \hline
    cis & c$\sharp$ \\ \hline
    cisis & c$\sharp\sharp$ \\ \hline
    des & d$\flat$\\ \hline
    d & d \\ \hline
    dis & d$\sharp$ \\ \hline
    es & e$\flat$\\ \hline
    e & e \\ \hline
    eis & e$\sharp$ \\ \hline
    fes & f$\flat$\\ \hline
    f & f \\ \hline
    fis & f$\sharp$ \\ \hline
    fisis & f$\sharp\sharp$ \\ \hline
    ges & g$\flat$\\ \hline
    g & g \\ \hline
    gis & g$\sharp$ \\ \hline
    gisis & g$\sharp\sharp$ \\ \hline
    as & a$\flat$\\ \hline
    a & a \\ \hline
    ais & a$\sharp$ \\ \hline
    b & h$\flat$\\ \hline
    h & h \\ \hline
    his & h$\sharp$ \\
    \hline
\end{tabular}
\end{center}

\subsection{\texttt{`WartosciNut'}}
Elementami tego typu są wartości (inaczej: długości trwania) nut. Wartości tego wyliczenia to liczby rzeczywiste. Na potrzeby niniejszego projektu ograniczono się do: 
\begin{center}
    \begin{tabular}{|c|c|}
        \hline
        \textbf{Nazwa wartości} & \textbf{Wartość liczbowa} \\
        \hline
        (cała) nuta & 4.0\\ \hline
        półnuta z kropką & 3.0 \\ \hline
        półnuta & 2.0 \\ \hline
        ćwierćnuta z kropką & 1.5 \\ \hline
        ćwierćnuta & 1.0\\ \hline
        ósemka & 0.5 \\ \hline
    \end{tabular}
    \end{center}

\subsection{\texttt{`Metrum'}}
Elementami tego typu są metra (metrum), w których mogą występować partytury. Na potrzeby projektu ograniczono się do:
\begin{itemize}
    \item 3/4
    \item 4/4
\end{itemize}

\subsection{\texttt{`Funkcje'}}
Elementami tego typu są funkcje harmoniczne (inaczej: rodzaje akordów). Na potrzeby tego typu ograniczono się do takich funkcji, jak:
\begin{itemize}
    \item Tonika (moll Tonika)
    \item Subdominanta (moll Subdominanta)
    \item Dominanta
    \item Dominanta septymowa
    \item Błąd - element zwracany wówczas, gdy niemożliwe jest przyporządkowanie danych dźwięków do żadnego, przewidzianego w projekcie, typu.
\end{itemize}

\subsection{\texttt{`Przewroty'}}
Elementami tego typu są wszystkie możliwe przewroty akordów, w uwzględnionych w projekcie funkcjach, a zatem: 
\begin{itemize}
    \item Postać zasadnicza 
    \item Przewrót pierwszy - tercja w basie
    \item Przewrót drugi - kwinta w basie
    \item Przewrót trzeci - septyma w basie. występuje tylko w przypadku dominanty septymowej
    \item Nie zdefiniowano - gdy podano błędną funkcję, nie można określić przewrotu.
\end{itemize} 

\subsection{\texttt{`ZdwojonySkladnik'}}
Jest to typ wyliczeniowy zwracany przez metodę \texttt{akord.ustal\_dwojenie()}. Może przyjąć 4 wartości:
\begin{itemize}
    \item \texttt{PRYMA = 0}
    \item \texttt{TERCJA = 1}
    \item \texttt{KWINTA = 2}
    \item \texttt{BRAK = 3}
\end{itemize} 
Nazwa elementu informuje o tym, który ze składników jest zdwojony.


\subsection{\texttt{`BezwzgledneKodyDzwiekow'}}
Elementami tego typu są nazwy dźwięków występujących w tonacji C-dur (bardziej obrazowo - białe klawisze fortepianu). Działanie typu opisano szczegółowo \hyperref[punkt:bezwzglednyKodDzwieku]{tutaj}. 

\subsection{\texttt{`DzwiekiWSkalach'}}
Jest to typ wyliczeniowy wykorzystywany przez metodę \texttt{sprawdzarka.czy\_glosy\_w\_swoich\_skalach}. Może przyjąć 3 wartości:
\begin{itemize}
    \item Poniżej skali (dźwięk znajduje się poniżej dolnej granicy skali) (\texttt{PONIZEJ\_SKALI = 0}) 
    \item W skali (\texttt{W\_SKALI = 1}) 
    \item Powyżej skali (\texttt{POWYZEJ\_SKALI = 2}) 
\end{itemize} 


\subsection{\texttt{`KrzyzowaniaGlosow'}}
Jest to typ wyliczeniowy wykorzystywany przez metodę \texttt{sprawdzarka.czy\_glosy\_nie\_skrzyzowane}. Może przyjąć 6 wartości:
\begin{itemize}
    \item \texttt{SOPRAN\_I\_ALT = 12}
    \item \texttt{SOPRAN\_I\_TENOR = 13}
    \item \texttt{SOPRAN\_I\_BAS = 14}
    \item \texttt{ALT\_I\_TENOR = 23}
    \item \texttt{ALT\_I\_BAS = 24}
    \item \texttt{TENOR\_I\_BAS = 34}
\end{itemize} 
Nazwa elementu typu mówi o tym, które dwa głosy zostały skrzyżowane. Wartość przypisana elementowi ilustruje poszczególne głosy - jeśli występuje 1, to z czymś skrzyżowany jest sopran (analogicznie alt, tenor i bas).\\
Jeśli sopran jest skrzyżowany z altem, to alt jest też skrzyżowany z tenorem, dlatego wystarcza jedynie połowa z 12 możliwych krzyżowań.


\subsection{\texttt{`NiepoprawneStopnie'}}
Jest to typ wyliczeniowy wykorzystywany przez metodę \texttt{sprawdzarka.czy\_glosy\_sa\_stopniami\_tonacji}. Może przyjąć 4 wartości:
\begin{itemize}
    \item \texttt{SOPRAN = 1}
    \item \texttt{ALT = 2}
    \item \texttt{TENOR = 3}
    \item \texttt{BAS = 4}
\end{itemize} 
Nazwa elementu informuje o tym, który ze składników zawiera niepoprawny (niebędący stopniem w danej tonacji) dźwięk.


\end{document}
