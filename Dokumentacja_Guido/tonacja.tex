\documentclass[dokumentacja.tex]{subfiles}

\begin{document}
\section{Tonacja}

\subsection{Zależności}
\paragraph*{Klasa \texttt{`Tonacja'} nie importuje żadnych innych klas.}

\subsection{Pola}
\begin{itemize}
    \item \texttt{\_nazwa: str} - przechowuje nazwę tonacji
    \item \texttt{\_czy\_dur: bool} - przechowuje prawdę, jeśli tonacja jest durowa, lub fałsz, jeśli jest molowa.
    \item \texttt{\_nazwy\_dzwiekow\_tonacji} - lista, w której znajdują się wszystkie stopnie właściwe dla pewnej tonacji
    \item \texttt{\_lista\_akordow} - przechowuje listę akordów w partyturze
\end{itemize}

\paragraph*{W klasa \texttt{`Tonacja'} przechowywane są również stałe:}
\begin{itemize}
    \item \texttt{\_WSZYSTKIE\_DUROWE\_TONACJE} - lista wszystkich tonacji durowych znajdujących się w kole kwintowym
    \item \texttt{\_WSZYSTKIE\_MOLOWE\_TONACJE} - lista wszystkich tonacji molowych znajdujących się w kole kwintowym
    \item \texttt{\_SLOWNIK\_DZWIEKOW\_DUROWE} - słownik, w którym każdej tonacji z powyższej listy tonacji durowych przypisano właściwe jej stopnie
    \item \texttt{\_SLOWNIK\_DZWIEKOW\_MOLOWE} - słownik, w którym każdej tonacji z powyższej listy tonacji molowych przypisano właściwe jej stopnie
\end{itemize}

\subsection{Metody}
\begin{itemize}
    \item Konstruktor parametryczny
        \begin{python}
    def __init__(self, nazwa_tonacji: str):
        \end{python}
        Konstruktor wypełnia pola klasy w zależności od podanej nazwy tonacji. W przypadku błędnej nazwy, podnoszony jest \texttt{ValueError}.

    \item Akcesor nazwy:
          \begin{python}
    def podaj_nazwe(self) -> str:
          \end{python}

    \item Akcesor trybu:
        \begin{python}
    def podaj_tonacje(self) -> tonacja.Tonacja:
        \end{python}

    \item Akcesor listy dźwięków:
    \begin{python}
    def podaj_liste_nazw_dzwiekow(self) -> list[str]:
    \end{python}
\end{itemize}
\end{document}
